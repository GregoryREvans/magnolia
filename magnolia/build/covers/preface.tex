\documentclass[11pt]{article}
\usepackage{fontspec}
\usepackage[utf8]{inputenc}
\setmainfont{STIXGeneral}
\usepackage[paperwidth=8.5in,paperheight=11in,margin=1in,headheight=0.0in,footskip=0.5in,includehead,includefoot,portrait]{geometry}
\usepackage[absolute]{textpos}
\TPGrid[0.5in, 0.25in]{23}{24}
\parindent=0pt
\parskip=12pt
\usepackage{nopageno}
\usepackage{graphicx}
\graphicspath{ {./images/} }
\usepackage{amsmath}
\usepackage{tikz}
\newcommand*\circled[1]{\tikz[baseline=(char.base)]{
            \node[shape=circle,draw,inner sep=1pt] (char) {#1};}}

\begin{document}

\begin{textblock}{23}(0, 1)
\begin{center}
\huge FOREWORD
\end{center}
\end{textblock}

\vspace*{0.25\baselineskip}

\begingroup
\begin{center}
\leftskip1.5in
In the front yard of the house where I grew up are two while magnolia trees which always seem to bloom too early.
\rightskip\leftskip
\phantom{text} \hfill (G.R.E.)
\end{center}
\endgroup

\vspace*{0.25\baselineskip}

\begingroup
\begin{center}
\leftskip1in
Lorem ipsum dolor sit amet, consectetur adipiscing elit. Morbi neque nibh, efficitur a eros at, elementum mollis nunc. Proin consectetur magna id purus euismod tristique. Aenean sit amet pulvinar nibh. Vivamus malesuada finibus pretium. Ut ligula mauris, commodo ac magna at, vestibulum accumsan purus. Vivamus tincidunt accumsan cursus. Ut arcu diam, vestibulum eget eros ultricies, sollicitudin facilisis tortor. Quisque massa arcu, dictum eget tristique sit amet, sagittis non nisl. Mauris varius lectus non nisl vehicula, ut feugiat ante accumsan. Nunc venenatis dui consequat erat porttitor, sit amet mollis nisl pharetra. Morbi hendrerit fringilla mauris, vitae rhoncus ante facilisis sit amet. Aenean erat felis, lobortis id dui sed, tempor finibus leo.
\rightskip\leftskip
\phantom{text} \hfill (G.R.E.)
\end{center}
\endgroup

\vspace*{1.25\baselineskip}

\begin{center}
\huge PERFORMANCE NOTES
\end{center}

\begingroup
\begin{center}

\leftskip0.5in
\pmb{Pitch} : \circled{1} Accidentals apply only to the pitch which they immediately precede, but persist through ties. \circled{2} Where no trilled pitch is notated, all trills should be performed as minor seconds.
\rightskip\leftskip
\phantom{text} \hfill \phantom{()}

\vspace*{0.25\baselineskip}

\leftskip0.5in
\pmb{Alternate Timbres} : Bisbigliando is notated as a trill preceded by the abbreviation \textit{Bis.} while rhythmicized timbre alterations are notated as a circled number above a note (such as \circled{1}, \circled{2}, or \circled{3}), where higher numbers refer to a greater deviation in timbre and pitch.
\rightskip\leftskip
\phantom{text} \hfill \phantom{()}

\vspace*{0.25\baselineskip}

\leftskip0.5in
\pmb{Multiphonics} : The included fingering charts are derived from Marcus Weiss's \textit{Die Spieltechnik des Saxophones}. The written pitches are desired and alternate fingerings may be employed as necessary.
\rightskip\leftskip
\phantom{text} \hfill \phantom{()}

\vspace*{0.25\baselineskip}

\leftskip0.5in
\pmb{Miscellaneous} : \circled{1} Triangle note heads indicate \textit{pizzicato}. \circled{2} Slash noteheads indicate a tone color that contains mostly air sound and some pitch while \circled{3} the hybrid notehead indicates equal parts air and pitch. \circled{4} The performer is at liberty to slur any notes with the exception of those with distinct articulations.
\rightskip\leftskip
\phantom{text} \hfill \phantom{()}
\end{center}
\endgroup

\vspace*{17\baselineskip}

\begin{center}
c. 2'45''
\end{center}

\end{document}
